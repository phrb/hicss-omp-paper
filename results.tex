\section{Results}
\label{sec:results}

After the students submitted the assignment for evaluation, we analyzed the reports and tutorials where errors were found. One of the examples shown in class presented an error, and multiple errors were found in several tutorials available on the Internet. Table~\ref{table:tutorials:with:problems} summarizes the errors found in tutorials from the Internet and their corrections by the students.
\begin{table}[ht]
\renewcommand{\arraystretch}{1.3}
\caption{Tutorials that presented some problem}
\label{table:tutorials:with:problems}
\centering
\begin{tabular}{|>{\centering\arraybackslash} m{.2cm}|>{\centering\arraybackslash} m{1.7cm}|  >{\centering\arraybackslash} m{2.75cm}|>{\centering\arraybackslash} m{2.75cm} |}
\hline
\textbf{Id}&\textbf{Tutorial} & \textbf{Identified Problem} & \textbf{Correction}\\
\hline

\begin{minipage}[m]{0.02\columnwidth}%
\centering
1
\end{minipage}&
\begin{minipage}[t]{0.22\columnwidth}%
\texttt{Tutorial de OpenMP C/C++~\cite{LCCV:UFAL:Tutorial}}
\end{minipage}&
\begin{minipage}[t]{0.33\columnwidth}%
Incorrect result. Race condition in results variables was detected.
\end{minipage}& 
\begin{minipage}[t]{0.33\columnwidth}%
The clause firstprivate was used to transform $aux\_dot$ in private variable in threads. \tiny{\texttt{\#pragma omp parallel firstprivate(aux\_dot)}}
\end{minipage}
\tabularnewline
\hline

\begin{minipage}[m]{0.02\columnwidth}%
\centering
2
\end{minipage}&
\begin{minipage}[t]{0.22\columnwidth}%
\texttt{Guide into OpenMP~\cite{OpenmpAndFork}}
\end{minipage}&
\begin{minipage}[t]{0.33\columnwidth}%
\texttt{OpenMP} interaction with Linux fork implementation using the GCC. The created thread pool are passed to child processes, but the fork() do not duplicate the threads causing deadlock in the child processes.
\end{minipage}&
\begin{minipage}[t]{0.33\columnwidth}%
Make fork() before application of pragmas in the parent process. Organize the code for child processes create their own thread pool.
\end{minipage}
\tabularnewline
\hline

\begin{minipage}[m]{0.02\columnwidth}%
\centering
3
\end{minipage}&
\begin{minipage}[t]{0.22\columnwidth}%
\texttt{Code Project~\cite{CodeProject}}
\end{minipage}&
\begin{minipage}[t]{0.33\columnwidth}%
Sintax error. Missing the~\lq\lq\{\rq\rq~after directive for.
\end{minipage}&
\begin{minipage}[t]{0.33\columnwidth}%
The students tested take off~\lq\lq\}\rq\rq~at the final causing segmentation fault error.
So they put the~\lq\lq\{\rq\rq~after the directive for.
\end{minipage}
\tabularnewline
\hline

\begin{minipage}[m]{0.02\columnwidth}%
\centering
4
\end{minipage}&
\begin{minipage}[t]{0.22\columnwidth}%
\texttt{Parallel Processing with OpenMP~\cite{BostonUniversity}}
\end{minipage}&
\begin{minipage}[t]{0.33\columnwidth}%
Error on use of ordered clause. It is necessary to indicate that the ordered clause will be used inside of loop.
\end{minipage}& 
\begin{minipage}[t]{0.33\columnwidth}%
The ordered clause was put.
\end{minipage}
\tabularnewline
\hline
\begin{minipage}[m]{0.02\columnwidth}%
\centering
5
\end{minipage}&
\begin{minipage}[t]{0.22\columnwidth}%
\texttt{Task Parallelism in OpenMP~\cite{UniversityUtah}}
\end{minipage}&
\begin{minipage}[t]{0.33\columnwidth}%
The $tid$ is not defined as private. It is shared by all threads the vector $A$ values are not calculated.
\end{minipage}&
\begin{minipage}[t]{0.33\columnwidth}%
Declare $tid$ as private using the \texttt{private(tid)} clause.
\end{minipage}
\tabularnewline
\hline
\begin{minipage}[m]{0.02\columnwidth}%
\centering
6
\end{minipage}&
\begin{minipage}[t]{0.22\columnwidth}%
\texttt{Introduction to Parallel Processing. OpenMP~\cite{UniversityEmory}}
\end{minipage}&
\begin{minipage}[t]{0.33\columnwidth}%
Use of \texttt{cout} function inside parallel region without critical clause. The threads are interrupted during the printing, messages are concatenated.
\end{minipage}&
\begin{minipage}[t]{0.33\columnwidth}%
The code was modified using the critical clause.
\end{minipage}
\tabularnewline
\hline
\begin{minipage}[m]{0.02\columnwidth}%
\centering
7
\end{minipage}&
\begin{minipage}[t]{0.22\columnwidth}%
\texttt{Introduction to Parallel and Distribution Systems. OpenMP Overview~\cite{UniversityWayne}}
\end{minipage}&
\begin{minipage}[t]{0.33\columnwidth}%
In the matrix multiplication code the variables $j$ and $k$ are shared by all threads generating wrong result.
\end{minipage}&
\begin{minipage}[t]{0.33\columnwidth}%
Addition of private clause on \tiny{\texttt{\#pragma omp parallel for private(j, k)}}
\end{minipage}
\tabularnewline
\hline
\begin{minipage}[m]{0.02\columnwidth}%
\centering
8
\end{minipage} &
\begin{minipage}[t]{0.22\columnwidth}%
\texttt{Introduction to OpenMP. Dijkstra sample~\cite{UniversityCalifornia}}
\end{minipage} & 
\begin{minipage}[t]{0.33\columnwidth}%
When the code is executed with more of 3 threads produces wrong results. There is an error in assigning values for $startv$ and $endv$ variables that keep the start and end positions of the shortest path. With this problem the algorithm does not reach all graph vertices. On line 88 a critical section is created for updating the minimum value and the vertex ($md$ and $mv$) but the code is not avoiding that updates occurs before another thread update the global vector, it is printing the values set in the initialization vector.
\end{minipage}&
\begin{minipage}[t]{0.33\columnwidth}%
The definition of variables $startv$ and $endv$ was modified using the original idea of algorithm. The \texttt{\#pragma omp barrier} directive was added after the \texttt{\#pragma omp critical}.
\end{minipage}
\tabularnewline
\hline

\end{tabular}
\end{table}

% Esses sites são os seguintes:
% http://bisqwit.iki.fi/story/howto/openmp/#OpenmpAndFork, 
% http://www.codeproject.com/Articles/60176/A-Beginner-s-Primer-to-OpenMP
% Universidade de Boston - www.bu.edu/tech/files/2011/09/openmp.ppt
% Tutorial disponiblizado pela Universidade de Utah, dado pelo seguinte link: http://www.cs.utah.edu/~mhall/cs4961f11/CS4961-L9.pdf.
% Tutorial da universidade de Emory (USA) OpenMP  http://www.mathcs.emory.edu/~cheung/Courses/355/Syllabus/91-pthreads/openMP.html
% Universidade do Estado de Wayne (USA) Apresentação de Cheng-Zhong Xu. site: http://www.ece.eng.wayne.edu/ czxu/ece561/lnotes/OpenMP.pdf
% Site de Question, Answer. http://askubuntu.com/questions/590755/openmp­unable­to­use­multiple­threads­no­error­a ssociated­with­compilation. Já tinha sido corregido.
% Universidade de California, Davis. Introduction to OpenMP.  http://heather.cs.ucdavis.edu/~matloff/openmp.html#tutorials. http://heather.cs.ucdavis.edu/~matloff/OpenMP/Examples/NM/Dijkstra.c

The \emph{first tutorial} in Table~\ref{table:tutorials:with:problems} has a problem caused by a confusion about which variables are shared and which should be private. This problem is similar to the ones identified by S\"{u}\ss~and Leopold~\cite{SuB:2005:CMO:1892830.1892863} as common correctness mistakes (\lq\lq{}Forgetting to mark private variables as such\rq\rq{}). It was the fourth most frequent mistake in that study.

\emph{Tutorial 2} presents a problem with the interaction between \texttt{OpenMP} and the system's OS and compiler. Specifically, the \texttt{GCC} \texttt{fork()} implementation. When we use the fork primitive, the Operating System creates other processes (child processes) duplicating all resources used by processes that execute the \texttt{fork()} call. Using \texttt{OpenMP} and creating child processes makes the thread pool created by \texttt{OpenMP} to be passed to child processes, but \texttt{fork()} does not duplicate the threads, which causes deadlock in the child processes.

The problem presented in \emph{tutorial 3} was a syntax error, there was a missing ~\lq\lq\{\rq\rq{} after the directive.

\emph{Tutorial 4}, presented in Table~\ref{table:tutorials:with:problems}, was found by students because of an error using the \texttt{ordered} clause. When using this clause it is necessary to indicate that it will be used inside a loop. A similar mistake involving the clause was identified as common by S\"{u}\ss~and Leopold~\cite{SuB:2005:CMO:1892830.1892863}. The opposite situation happened when an \texttt{ordered} clause was put into a \texttt{for} construct without being specified by a separate \texttt{ordered} clause inside the loop. This is a restriction on the nesting of regions, an \texttt{ordered} region must be closely nested inside a loop or parallel loop region with an \texttt{ordered} clause~\cite{openmp:api:2011}.

Other problems in private variable definition were found in \emph{tutorials 5} and \emph{7}. In \emph{tutorial 5} the \texttt{tid} variable is not defined as private, causing some values in the array not being calculated. In \emph{tutorial 7} the variables \texttt{j} and \texttt{k} in the matrix multiplication are shared by all threads, generating wrong results.

In \emph{tutorial 6} the \texttt{cout} function is called inside a parallel region without a protecting \texttt{critical} clause. The call becomes a critical section, because the threads are interrupted during the printing and messages are concatenated generating incorrect results. The code was modified by students using the \texttt{critical} clause to ensure that the call is executed by one thread at a time.

\emph{Tutorial 8} presents a synchronization problem that causes an error in the assignment of values for the \texttt{startv} and \texttt{endv} variables that keep the start and end positions of the shortest path. Because of this problem the algorithm does not reach all graph vertices. The critical section is created to update the minimum value and the vertex (\texttt{md} and \texttt{mv}), but the code is not avoiding the occurrence of updates before another thread updates the global array. If all vertices are not visited some initialized values are not updated during the program execution, which prints the incorrect values. The code was modified using the original algorithm idea and a barrier was added to ensure synchronization using the \texttt{\#pragma omp barrier} directive after the \texttt{\#pragma omp critical}.

Table~\ref{table:summary:of:errors} summarizes the type of errors found by the students in Internet tutorials.

\begin{table}[htpb]
\renewcommand{\arraystretch}{1.3}
\caption{Summary of errors}
\label{table:summary:of:errors}
\centering
\begin{tabular}{|@{$~$}l@{ }|@{$~$}l@{ }|@{$~$}l@{ }|}
\hline
\textbf{Kind} &\textbf{Quantity} & \textbf{Tutorial} \\
\hline

\begin{minipage}[m]{0.35\columnwidth}%
private-shared
\end{minipage} &
\begin{minipage}[t]{0.10\columnwidth}%
\centering
3
\end{minipage} & 
\begin{minipage}[t]{0.20\columnwidth}%
1, 5 and 7
\end{minipage}
\tabularnewline
\hline

\begin{minipage}[m]{0.35\columnwidth}%
syntax error
\end{minipage} &
\begin{minipage}[t]{0.10\columnwidth}%
\centering
1
\end{minipage} & 
\begin{minipage}[t]{0.20\columnwidth}%
3
\end{minipage}
\tabularnewline
\hline

\begin{minipage}[m]{0.35\columnwidth}%
wrong use of directive
\end{minipage} &
\begin{minipage}[t]{0.10\columnwidth}%
\centering
1
\end{minipage} & 
\begin{minipage}[t]{0.20\columnwidth}%
4
\end{minipage}
\tabularnewline
\hline

\begin{minipage}[m]{0.35\columnwidth}%
synchronization problems
\end{minipage} &
\begin{minipage}[t]{0.10\columnwidth}%
\centering
2
\end{minipage} & 
\begin{minipage}[t]{0.20\columnwidth}%
6 and 8
\end{minipage}
\tabularnewline
\hline

\end{tabular}
\end{table}

Comparing the problems found in the experiment to the most common \texttt{OpenMP} mistakes~\cite{SuB:2005:CMO:1892830.1892863} we can see how some errors are recurrent, for instance, the decision about isolating data using declarations of shared or private variables.