\begin{abstract}

Selecting the most effective tools and languages for teaching parallel and
distributed computing presents a challenge for teachers and students. Often a
high-level solution seems easier, but it comes with the cost of hiding
underlying concepts and problems. In this paper we present a comparison between
the \textit{Pthreads} and \textit{OpenMP} APIs considering the experiences and
performance results of undergraduate and graduate Computer Science students
from the University of São Paulo.
We show that students were able to achieve better performance results using
\textit{Pthreads} than \textit{OpenMP}, despite their perception that
\textit{Pthreads} is harder to learn and use.

\end{abstract}

%
% The code below should be generated by the tool at
% http://dl.acm.org/ccs.cfm
% Please copy and paste the code instead of the example below.
%

\begin{CCSXML}
<ccs2012>
 <concept>
  <concept_id>10010147.10010169</concept_id>
  <concept_desc>Computing methodologies~Parallel computing methodologies</concept_desc>
  <concept_significance>500</concept_significance>
 </concept>
 <concept>
  <concept_id>10010147.10010919</concept_id>
  <concept_desc>Computing methodologies~Distributed computing methodologies</concept_desc>
  <concept_significance>500</concept_significance>
 </concept>
</ccs2012>
\end{CCSXML}

\ccsdesc[500]{Computing methodologies~Parallel computing methodologies}
\ccsdesc[500]{Computing methodologies~Distributed computing methodologies}



\keywords{Parallel and Distributed Computing, OpenMP, Pthreads}

