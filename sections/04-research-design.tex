\section{Research Design}
\label{sec:researchdesign}

This section describes our research design. We devised a questionnaire
regarding student experience with solving the same problem using
\textit{Pthreads} and \textit{OpenMP} during the Concurrent, Parallel and
Distributed Computing Course at the Computer Science program of the University
of São Paulo. We answered 3 research questions using analyses of the students'
responses.

\subsection{The Concurrent, Parallel and Distributed Computing Course}

The Computer Science program of the University of São Paulo offers Concurrent,
Parallel and Distributed Computing courses for graduate and undergraduate
students in a single class. The course aims to promote meaningful exchanges of
knowledge and experience between undergraduate and graduate students.

The course presents students with a series of software engineering challenges
in the form of assignments. The final is composed of an essay and a
presentation on a relevant topic related to Concurrent, Parallel and
Distributed Computing.

We devised an assignment to measure the students' difficulties to
learn and use \textit{Pthreads} and \textit{OpenMP} to solve
a simple problem. Students had to implement parallel versions
of a sequential code using \textit{Pthreads} and \textit{OpenMP}.
We provided a sequential implementation for the calculation of
the Mandelbrot Set. The sequential code presented an embarrassingly
parallel problem with two nested loops, and a third loop with
dependent iterations that could not be parallelized easily.

The assignment required that the students measure the performance of the
sequential code and the performance improvements they achieved using
\textit{Pthreads} and \textit{OpenMP}.  After completing the assignment the
students were asked to answer a questionnaire.

\subsection{Questionnaire}

The questionnaire had two parts. The first part asked questions
about the students' previous experiences with common programming languages and
parallel and distributed programming concepts and tools. We were also
interested in students' perception of the impact of our classes on
\textit{Pthreads} and \textit{OpenMP} on their learning difficulty.

The questions of the second part were related to the students'
experience with \textit{Pthreads} and \textit{OpenMP} \textit{after}
the completion of the assignment.

We obtained responses, and consent to use them in our research, from 38 of the
54 students enrolled in the Concurrent, Parallel and Distributed Computing
course. Section \ref{sec:background} presents a detailed description of
each set of questions our analysis of the students' responses.

\subsection{Research Questions}

In previous work~\cite{goncalves:OpenMPNotEasy} we showed that
teaching \textit{OpenMP} requires teaching important parallel
and distributed computing concepts first, since it is easy
to find errors on commonly used \textit{OpenMP} learning material.

In this paper we argue that students are capable of learning and using
\textit{Pthreads} to obtain good performance improvements, despite their
perception that \textit{OpenMP} is much easier to learn and use.
The research questions that led the development of the questionnaire
and our analysis of the students' responses were the following:

\begin{description}
    \item[RQ1:] \textit{According to the students' perception, which API was
        easier to learn and use?}
    \item[RQ2:] \textit{With which API did the students improve performance the
        most?}
    \item[RQ3:] \textit{Did the students' performance improvements match their
        perception?}
\end{description}

To answer the first two questions we used students' responses to
our questionnaire. To answer the third question we analysed the
students' submissions and parallel implementations. The
students' responses and our analysis are presented next.
