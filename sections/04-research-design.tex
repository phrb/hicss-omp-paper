\section{Research Design}
\label{sec:researchdesign}

This section describes our research design. We devised a questionnaire
regarding student experience with solving the same problem using
\textit{Pthreads} and \textit{OpenMP} during the Concurrent, Parallel and
Distributed Computing Course at the Computer Science program of the University
of São Paulo. We answered 3 research questions analysing the students'
responses.

\subsection{The Course}

The Computer Science program of the University of São Paulo offers Concurrent,
Parallel and Distributed Computing courses for graduate and undergraduate
students in a single class. The course aims to promote meaningful exchanges of
knowledge and experience between undergraduate and graduate students.

The course presents students with a series of software engineering challenges
in the form of assignments involving GPU programming with CUDA and distributed
computing with OpenMPI. The final is composed of an essay and a presentation on
a relevant topic related to Concurrent, Parallel and Distributed Computing.
The course material is available
online~\footnote{\url{https://phrb.github.io/MAC5742-0219} [Accessed in
07/08/2017]}.

We devised an assignment to measure the students' difficulties to learn and use
\textit{Pthreads} and \textit{OpenMP} to solve a simple problem. Students had
to form groups of 2 to 3 members and implement parallel versions of a
sequential code using \textit{Pthreads} and \textit{OpenMP}.  We provided a
sequential implementation for the calculation of the Mandelbrot
Set~\cite{douady1984etude}. The sequential code presented an embarrassingly
parallel problem with two nested loops, and a third loop with dependent
iterations that could not be parallelized easily.

\begin{figure}[htb]
\begin{minipage}{\linewidth}
\begin{lstlisting}[language=C, basicstyle=\ttfamily\scriptsize, numbers=left,
                   frame=no, showspaces=false, showstringspaces=false,
                   numberstyle=\tiny,
                   xleftmargin=0.5cm,
                   keywords={%
                       DATATYPE, pthread_t, pthread_create,
                       pthread_join, task_function, NULL, int, main,
                       void, printf, return, pthread_mutex_t,
                       pthread_attr_t, pthread_attr_init,
                       MAX_THREADS, SIZE, char, struct, malloc,
                       MIN, pthread_mutex_lock, pthread_mutex_unlock,
                       pthread_exit, from, to, and%
                       },
                   otherkeywords={::, \#pragma, \#include, <<<,>>>, \&, \*, +, -, /, [, ], >, <}
                   ]
z = 0;
for(x from 0 to x_max - 1) {
    /* Independent iterations */
    for(y from 0 to y_max - 1) {
        /* Independent iterations */
        for (iteration from 0 to iteration_max,
             and f_c(z) < limit) {
            /* Iterations depend on
               previous values of z */
            z = f_c(z);
        }
    }
}
\end{lstlisting}
\end{minipage}
\caption{Pseudocode for the computation of the Mandelbrot set}
\label{lst:mandelbrot-pseudo}
\end{figure}

The Mandelbrot set is informally defined as the set of complex numbers $c$ for
which the function $f_c(z) = z^2 + x$ does not diverge when iterated starting
in $z=0$, that is, the sequence the sequence $f_c(0), f_c(f_c(0)),
f_c(f_c(f_c(0))),\dots$ is always limited. Figure~\ref{lst:mandelbrot-pseudo}
shows pseudocode for the computation of the Mandelbrot set, listing which loops
have dependent and independent iterations.

The assignment required that the students measure the performance of the
sequential code and the performance improvements they achieved using
\textit{Pthreads} and \textit{OpenMP}. After completing the assignment we asked
the students to answer a questionnaire.

\subsection{Questionnaire}

After students organized themselves in groups of 2 to 3 students and designed
and implemented their parallel versions of the code, each individual student
was asked to answer a questionnaire. The questionnaire had two parts. The first
part asked questions about the students' previous experiences with common
programming languages and parallel and distributed programming concepts and
tools. We were also interested in students' perception of the impact of our
classes on \textit{Pthreads} and \textit{OpenMP} on their learning difficulty.

The questions of the second part were related to the students'
experience with \textit{Pthreads} and \textit{OpenMP} after
the completion of the assignment.

We obtained the ethical approval from the Department of Computer Science at the
University of São Paulo to conduct this study based on the questionnaire. We
collected responses, and consent to use them in our research, from 38 of the 54
students enrolled in the Concurrent, Parallel and Distributed Computing course.
Section \ref{sec:responses} presents a detailed description of each set of
questions and our analysis of the students' responses.

\subsection{Research Questions}
\label{sec:resques}

In previous work~\cite{goncalves:OpenMPNotEasy} we showed that teaching
\textit{OpenMP} requires the introduction of important parallel and distributed
computing concepts first, since it is easy to find errors on commonly used
\textit{OpenMP} learning material.

In this paper we argue that students are capable of learning and using
\textit{Pthreads} to obtain good performance improvements, despite their
perception that \textit{OpenMP} is much easier to learn and use.
The research questions that led the development of the questionnaire
and our analysis of the students' responses were the following:

\begin{description}
    \item[RQ1:] \textit{According to the students' perception, which API was
        easier to learn and use?}
    \item[RQ2:] \textit{With which API did the students improve performance the
        most?}
    \item[RQ3:] \textit{Did the students' performance improvements match their
        perception?}
\end{description}

To answer the first two questions we used students' responses to
our questionnaire. To answer the third question we analysed the
students' submissions and parallel implementations. The
students' responses and our analysis are presented next.
