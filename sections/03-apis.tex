\section{OpenMP \& Pthreads}
\label{sec:apis}

In this section we discuss the main concepts involved in both OpenMP and
pthreads libraries as they can both be used to implement parallel programs by
using the fork-join model, on which the sequential region execution is followed
by parallel regions execution.


\subsection{Pthreads}

The pthreads library specification was design to solve the problem of having
multiple hardware vendors delivering specific APIs and tools for developing
multithread programs that were not portable across multiple platforms. The
thread programming interface has been specified by the IEEE POSIX 1003.1c
standard and any implementation adhering to this standard is called POSIX
threads or pthreads.

In C/C++ languages the pthreads library is used to control thread execution
on GNU/Linux by providing mechanisms to create and synchronize threads. In this
model, the execution starts in a sequential region and threads needed to be
explicitly started in order to start a parallel region.

Listing \ref{lst:listing-pthreads} presents a sample of code that uses the
pthreads library. As explained before, the program starts its execution in a
single thread and up to line 6 only one thread is executing the main flow of
instructions.

When the \texttt{pthread\_create()} function is called, a new thread is created
and then the application enters a parallel region with two threads executing.
The main mechanism available for synchroninzing thread executions are thread
barrier. The pthread library implements barriers at the
\texttt{pthread\_join()} where the thread that reaches that point will wait for
the completion of other threads to continue the execution of the main flow.
operation. In line 10 we can see an example of thread synchronization using the
barrier definition.

\begin{lstlisting}[language=C, basicstyle=\footnotesize, numbers=left, 
                   frame=single, showspaces=false, showstringspaces=false,
                   caption={Pthread Sample Program}, captionpos=b,
                   label=lst:listing-pthreads]
int main(){
    pthread_t thread1 ;
    DATA TYPE *params = 10;
    int iret1;
    
    iret1 = pthread_create(&thread1 , NULL, 
                           task function , 
                           (void*) params);
    printf ("Thread 1 return : %d\n", iret1 ) ;
    pthread_join(thread1 , NULL) ;

    return 0;
}
void *task function(void *ptr) { 
    DATATYPE *value;
    value = (DATA TYPE *) ptr ;
    /* Computing */
}
\end{lstlisting}

The choice to use the pthreads library, and not OpenMP, can be motivated by the
fact that pthreads is a common library, available on platforms with GNU/Linux,
and operates in a lower level of abstraction in comparison with OpenMP threads.

\subsection{OpenMP}

OpenMP is an specification that defines a set of compiler directives, libraries
and environment variables with the goal to deliver tools that help exploring
the inherent program parallelism by means of threads. The OpenMP specification
is kept by a group named OpenMP Architecture Review Board \footnote{} composed
by several hardware manufacturers as well as parallel software developers and
users.

The code structure of an OpenMP program is composed by intertwined sequential
and parallel regions that always start with a sequential one, i.e., the program
starts with a sequential region - the master thread, then a series of parallel
ones starts and ends - fork-join model.

The OpenMP directives are implemented in C/C++ and Fortran compilers. At the
C/C++ compilers, the specification of parallel executions is done by means of
\textit{pragma} directives. The pragma directives are defined inside the code
by means of the \texttt{\#pragma} keyword. In case the compiler does not
recognize the keyword, the compilation procedure is not affected and the
sequential version of the program is built.

The directives format follow the rule \texttt{\#pragma directive [clauses]}.
Each line has at least one directive and may contain one or more clauses. The
main directives available at the specification are: 
\begin{itemize}
    \item Work sharing: this set of directives dictate how the work is shared
    among threads;
    \item Tasks: define independent code regions that can be executed by any
    thread available. It also defines the dependencies between tasks;
    \item Synchronization: used to synchronize the code regions that are being
    executing within separated threads.
\end{itemize}

Listing \ref{lst:listing-omp} shows a basic OpenMP example. The example only
shows the declaration of a parallel region named \texttt{firstprivate} (line
1), then a region that should only be executed by a single thread (line 2),
then a for that should be executed in parallel and that the work division will
bu automatic (line 6). Then, in line 10 we define a critical region which can
only be executed by a single thread at a time. In the end, at line 12 we define
a region that should only be executed by the master thread.


\begin{lstlisting}[language=C, basicstyle=\footnotesize, numbers=left, 
                   frame=single, showspaces=false, showstringspaces=false,
                   caption={OpenMP Sample Program}, captionpos=b,
                   label=lst:listing-omp]
#pragma omp parallel firstprivate (aux dot){ 
    #pragma omp single
    printf ("Begin of the parallel region, 
            number of threads : %d\n", 
            omp get num threads ()) ;
    #pragma omp for schedule (auto) 
    for(i = 0; i < SIZE; i++){
        aux dot += A[i] * B[i]; 
    }    
    #pragma omp critical
    dot += aux_dot ;
    #pragma omp master
    printf("Final result: %d.\n", dot);
}
\end{lstlisting}
