\section{Background}
\label{sec:background}

%\todo[inline,color=cyan,author=Pedro]{Cite our previous paper, making it clear that we are the authors}
%\todo[inline,color=cyan,author=Pedro]{Describe related work}
%\todo[inline,color=cyan,author=Pedro]{Describe related work since last paper}

The discipline of Parallel and Distributed Computing has evolved from an
elective topic into one of the main curricular components in a Computer Science
courses, as pointed out in \cite{acmcurricula}, thus, deciding what
technology/framework to teach students in Parallel Computing class has a great
impact on the student's ability to develop efficient solutions for algorithmic
problems. 

There are several chain of thoughts on what is the best level of abstraction to
teach parallel programming courses. In \cite{6565518}, Falcao argues that
introducing a higher level parallel programming interface such as OpenMP in the
beginning of a CS course would cause very little overhead and would bring
great benefits for the rest of the course. The approach of introducing mostly
practical learning experiences spread all over a CS course - increasing in
depth as the student advances in classes was also supported by the authors in
\cite{FOLEY2017138}. In their work they introduce a tool named OnRamp, a web
portal that aims to help students learn how to write parallel code in a
top-down approach, i.e., from practical experiences to more conceptual ones.

Pllana et al. use a similar approach as described in \cite{Pllana:2009} where
the authors adopt a Model Driven Development to build parallel solutions by
means of Parallel Building Blocks. The authors describe a programming
environment that takes these high level model and transform them into source
code. In that way, the tool goal is to hide the complexity of writing parallel
code away from the students.

In contrast to these top-down approaches for teaching parallel programming
there is the bottom-up approach on which, learning a software tool is just one
of the objectives of a parallel programming class and that would come after the
student are presented the concepts of parallel programming.

In a previous investigation \cite{goncalves:OpenMPNotEasy}, we showed that
first teaching the conceptual aspects of parallel programming instead of
jumping into teaching higher level parallel programming interfaces such as
OpenMP is of extreme importance mainly because the available
resources/tutorials for learning such APIs may contain errors that would be
otherwise difficult to find. In our work we listed several sources of
tutorials/hands-on materials retrieved from the web on which under-grad
students were able to point parallel correctness mistakes
\cite{SuB:2005:CMO:1892830.1892863}.

Another point that should be taken into consideration is that, not all
parallel solutions can be expressed using tools such as OpenMP. This is the
main topic discussed by Sü\ss and Leopold in \cite{Leopold:userOpenMP}, where
they show that, by using a simple sorting algorithm as example, we can identify
patterns that are difficult to capture using OpenMP and that could be easily
handled using a lower level programming interface such as POSIX Threads.

In \cite{ADAMS201731}, Adams also showed a set of patterns implemented using
PThreads that help students understand some aspects of parallel programming
that would be harder to grasp by using only OpenMP examples. His catalog of
patterns also contains many solutions implemented using OpenMP.

In this paper we argue that teaching a lower level parallel programming
interface such as PThreds is equally important to teaching higher level
interfaces such as OpenMP and both should be part of a complete parallel
programming class curriculum like the one described by Brown et al. in
\cite{Brown:2010:SPC:1971681.1971689}, where it has been pointed that the
choice of a software tool should be only one of the knowledge areas that should
be explored during a parallel programming course, along with data structure and
algorithms; software design; and parallel hardware platforms.

\todo[inline,color=orange,author=Raphael]{Maybe we should make a better
connection between the end of this section and the next one;}
