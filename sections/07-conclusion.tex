\section{Conclusion}
\label{sec:conclusion}

In this paper we elaborated a questionnaire to gauge students' perceptions and
measure their difficulty to learn \textit{OpenMP} and \textit{Pthreads}.  The
questionnaire contained questions that regarded each technology individually
and in comparison with the other.

The students' responses to the questionnaire enable the conclusion that the
students perceived less difficulty using and learning \textit{OpenMP}, in the
context of a course assignment, but that they also achieved greater performance
improvements using \textit{Pthreads}.

Contrary to our analysis, students reported the perception that using
\textit{OpenMP} made it easier to improve the performance of the sequential
code we provided.  Our analysis of their submissions revealed that most 
of the students ($52.2\%$) achieved better performance in the assignment with
either \textit{OpenMP} ($17.4\%$) or \textit{Pthreads} ($34.8\%$).
Contradicting the perception of the students, the implementation using
\textit{Pthreads} achieved better performance in twice the submissions.

The perceptions of usability and performance improvement regarding
\textit{OpenMP} was not reflected in the actual implementations and
submissions. We assert that \textit{OpenMP} is still not as easy as it appears
to be, and that the students' responses strengthen the argument for teaching
lower-level technologies for parallel and distributed computing.
The main contribution of this paper is the conclusion that students
are able to achieve better performance results using \textit{Pthreads}
than using \textit{OpenMP}, despite their perception that \textit{Pthreads}
is harder to learn and use.

In our study we have not compared the material used for teaching both
technologies in depth. In particular, we have not made any comparison regarding
the difficulty associated with the notes used for teaching both technologies.
Thus, on the matter of threats to the study validity, the students' perceptions
of the difficulty level of the references, used during the class, were not
taken into account. Also, in order to provide a more in-depth statistical
analysis, the study needs to be performed in several classes of the Concurrent,
Parallel and Distributed Computing courses.  Nevertheless, this pilot study
presents some great insights that will help devise statistical hypothesis that
should be tested in later studies.

Brown \textit{et al.}~\cite{Brown:2010:SPC:1971681.1971689} points that the
choice of a software tool should be only one of the topics explored during a
parallel programming course.  In future work we will compare the perception and
performance of students regarding learning and achieving performance
improvements using different languages and tools for parallel and distributed
computing.  Since we only studied an embarrassing parallel problem which could
give an initial advantage to \textit{OpenMP}, in future work we will explore
how more difficult parallel and distributed problems impacts the students'
perceptions and performance.
