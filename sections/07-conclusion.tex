\section{Conclusion}
\label{sec:conclusion}

In this paper we elaborated a questionnaire to gauge students' perceptions and
measure their difficulty to learn \textit{OpenMP} and \textit{Pthreads}.  The
questionnaire contained questions that regarded each technology individually
and in comparison with the other.  The students were enrolled in the
Concurrent, Parallel and Distributed course from the University of São Paulo,
Brazil.

The students' responses to the questionnaire enabled conclude that the students
perceived less difficulty using and learning \textit{OpenMP}, in the context of
a course assignment, but that they also achieved greater performance
improvements using \textit{Pthreads}.

Contrary to our analysis, students reported the perception that using
\textit{OpenMP} made it easier to improve the performance of the sequential
code we provided.  Our analysis of their submissions revealed that the majority
of the students ($52.2\%$) achieved better performance in the assignment with
either \textit{OpenMP} ($17.4\%$) or \textit{Pthreads} ($34.8\%$).
Contradicting the perception of the students, the implementation using
\textit{Pthreads} achieved better performance in twice the submissions.

The perceptions of usability and performance improvement regarding
\textit{OpenMP} was not reflected in the actual implementations and
submissions.  We assert that \textit{OpenMP} is still not as easy as it appears
to be, and that the students' responses strengthen the argument for teaching
lower-level technologies for parallel and distributed computing.
The main contribution of this paper is the conclusion that students
are able to achieve better performance results using \textit{Pthreads}
than \textit{OpenMP}, despite their perception that \textit{Pthreads}
is harder to learn and use.

In future work we will compare the perception and performance of students
regarding learning and achieving performance improvements using different
languages and tools for parallel and distributed computing. We will also study
how attempting solving more difficult parallel and distributed problem impacts
the students' perceptions and performance.
