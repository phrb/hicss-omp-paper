\section{Introduction}
\label{sec:introduction}

\todo[inline,color=cyan,author=Pedro]{Write new introduction}

The selection of teaching tools for Parallel and Distributed courses impacts
the development of students' abilities to solve algorithmic problems
efficiently. The importance of selecting good tools for teaching Parallel and
Distributed Computing has been increasingly important, especially since the
topic became a core component of the ACM undergraduate computer science
curricula in 2013~\cite{acmcurricula}.



In this work we have gathered data from assignments made by graduate students
for the \emph{Introduction to Parallel and Distributed Computing} course. Among
other exercises, the students were asked to ...


In this paper we present data that supports the importance of composing a
curriculum for Parallel and Distributed Computing courses using low-level
programming interfaces, such as \textit{Pthreads}, as well as high-level
interfaces, such as \textit{OpenMP}.  Brown \textit{et
al.}~\cite{Brown:2010:SPC:1971681.1971689}, points that the choice of a
software tool should be only one of the topics explored during a parallel
programming course. Topics such as data structures, algorithms, software design
and parallel hardware platforms should also be explored.

