\section{Introduction}
\label{sec:introduction}

Multi-core architectures became essential and ubiquitous for High-Performance
Computing in the last decade, their relevance increased by recent slowdowns on
Moore's law and the end of Dennard scaling \cite{esmaeilzadeh2012dark}.
Selecting the appropriate tool for programming multi-core architectures is also
essential, since it is necessary to program a multi-core architecture
effectively to leverage its performance.

Programming multi-core architectures demand different skills and tools from
what is needed for traditional sequential programming. Selecting an effective
language, framework or API for parallel and distributed programming is a
challenge for teachers and students, especially due to the quantity and
heterogeneity of available resources.

Tools with a higher level of abstraction may provide a smoother transition from
sequential to parallel and distributed programming, making it easier for
students to develop working code. The cost of high-level abstractions often is
the hiding of underlying concepts, problems and solutions, as well as code
performance.

The contribution of this paper is a comparison between \textit{Pthreads} and
\textit{OpenMP}, two commonly used APIs for parallel and distributed
programming. The comparison considers the perceptions and performance results
from undergraduate and graduate students enrolled in the course of Concurrent,
Parallel and Distributed Computing of the University of São Paulo, Brazil.  We
show that students were able to achieve better performance results with
\textit{Pthreads} than with \textit{OpenMP}, despite their perception that
\textit{Pthreads} is harder to learn and use.

The paper is organized as follows.  Section \ref{sec:apis} briefly describes
\textit{OpenMP} and \textit{Pthreads}.  Section \ref{sec:background} discusses
related work.  Section \ref{sec:researchdesign} presents the course of
Concurrent, Parallel and Distribute Computing of the University of São Paulo,
and introduces the questionnaire and our research questions.  Section
\ref{sec:responses} presents, discusses and analyses the students' responses to
the questionnaire.  Section \ref{sec:conclusion} summarizes the results and
presents future work.
