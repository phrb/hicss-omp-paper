\section{Pthreads \& OpenMP}
\label{sec:apis}

The selection of teaching tools for Parallel and Distributed courses impacts
the development of students' abilities to solve algorithmic problems
efficiently. The importance of selecting good tools for teaching Parallel and
Distributed Computing has been increasingly important, especially since the
topic became a core component of the ACM undergraduate computer science
curricula in 2013~\cite{acmcurricula}.  In this section we briefly describe
\textit{OpenMP} and \textit{Pthreads}, two commonly used tools for parallel and
distributed programming.

\subsection{Pthreads}

The \textit{Pthreads} library specification was designed to provide a standard
and portable API for developing multi-threaded programs for multiple vendor
platforms.  The programming interface was specified by the \textit{IEEE POSIX
1003.1c} standard. Any implementation adhering to this standard is said to be
POSIX threads, or \textit{Pthreads}.

The \textit{Pthreads} library controls thread execution by
creating and synchronizing threads. In this model, the execution begins in a
sequential region and threads must be explicitly started to begin a parallel
region.  Listing~\ref{lst:pthreads} presents a \textit{Pthreads}
example. The code shows the parallel computation of the dot product of two
arrays.

The program in Listing~\ref{lst:pthreads} runs in a single thread until
line 22, when \texttt{pthread\_create} is called.  Each call
in the loop initializes a new thread.
The parallelization in this example divides the arrays in chunks
and shares multiplications between threads.
One thread synchronization tool available in \textit{Pthreads} is the barrier,
implemented as calls to \texttt{pthread\_join}. A thread that reaches a barrier
waits for other threads. Line 27 shows an example of thread synchronization
using a barrier.
To avoid race conditions on the \texttt{dot} shared variable
the example uses \textit{mutexes} to create a critical
region, as seen in lines 42 and 44.

\begin{lstlisting}[language=C, basicstyle=\ttfamily\scriptsize, numbers=left,
                   frame=no, showspaces=false, showstringspaces=false,
                   caption={\textit{Pthreads} Sample Program}, captionpos=b,
                   numberstyle=\tiny,
                   xleftmargin=0.5cm,
                   label=lst:pthreads, keywords={%
                       DATATYPE, pthread_t, pthread_create,
                       pthread_join, task_function, NULL, int, main,
                       void, printf, return, pthread_mutex_t,
                       pthread_attr_t, pthread_attr_init,
                       MAX_THREADS, SIZE, char, struct, malloc,
                       MIN, pthread_mutex_lock, pthread_mutex_unlock,
                       pthread_exit%
                       },
                   otherkeywords={::, \#pragma, \#include, <<<,>>>, \&, \*, +, -, /, [, ], >, <}
                   ]
int A[SIZE], B[SIZE]; int dot = 0;
pthread_mutex_t lock;

int main(int argc, char *argv []){
  int num_threads, i, part_size;
  pthread_t tid [MAX_THREADS];
  pthread_attr_t attr;
  pthread_attr_init(&attr);
  /* Array initialization supressed */

  part_size = (SIZE + num_threads - 1) / num_threads;

  for (i = 0; i < num_threads; i++){
      struct arg_struct *args = \
              malloc(sizeof(struct arg_struct));

      args->id = i;
      args->i_part = (i * part_size);
      args->e_part = MIN(((i + 1) * (part_size)),
                         SIZE);

      pthread_create(&(tid[i]) ,&attr,
                     task_function,
                     (void *) args);
  }
  for (i = 0; i < num_threads; i++) {
      pthread_join(tid[i], NULL);
  }
  printf("Result: %d.\n", dot);
  return 0;
}

void *task_function (void *arguments){
  struct arg_struct *args = (arg_struct *) arguments;
  int i, partial_dot = 0;

  for(i = args->i_part; i < args->e_part; i++){
      partial_dot += A[i] * B[i];
  }

  /* Critical region */
  pthread_mutex_lock(&lock);
  dot += partial_dot;
  pthread_mutex_unlock(&lock);
  pthread_exit(0);
}
\end{lstlisting}

\subsection{OpenMP}

\textit{OpenMP} is an specification that defines a set of compiler directives,
libraries and environment variables that help exploring program parallelism.
The \textit{OpenMP} specification is kept by the \textit{OpenMP Architecture
Review Board}, composed by hardware manufacturers, and parallel software
developers and users.

\textit{OpenMP} implements the \textit{fork-join} model. An \textit{OpenMP}
program contains weaved sequential and parallel regions and always starts with
a sequential region, or master thread.  \textit{OpenMP} directives are
implemented in C/C++ and Fortran compilers. In C/C++ compilers the
specification of parallel executions is done using \textit{pragmas}, defined in
by using the \texttt{\#pragma} keyword.
The compilation process is not affected by the removal of the pragma,
which will result in the generation of a sequential version of the program.

\textit{OpenMP} directives follow the format \texttt{\#pragma directive
[clauses]}.  Each line contains at least one directive and may contain one or
more clauses. The main directives defined by the specification are related
to work sharing, task definition and dependencies, and synchronization.

Listing \ref{lst:omp} presents an \textit{OpenMP} example for the
same computation shown in Listing \ref{lst:pthreads}. The
declaration of a parallel region with the \texttt{firstprivate} pragma on line
1 indicates the variable \texttt{aux\_dot} should be private to every thread
and initialized with the enclosing scope's value.  Line 3 will be executed by
only one thread, because of the \texttt{single} pragma.
The loop in line 8 will be executed in parallel, with automatic work
division.  The pragma at line 12 defines a critical region which can only be
executed by a single thread at a time. The pragma at line 15 defines a
region that should only be executed by the master thread.

Note that the examples shown in Listings \ref{lst:pthreads} and
\ref{lst:omp} solve the same problem. The example in \textit{Phthreads}
contains much more code than the example in \textit{OpenMP}, and seems
to be much more complex as well. It is therefore understandable that students might
feel overwhelmed by \textit{Pthreads} example, especially when both examples are
presented side-by-side.

\begin{lstlisting}[language=C, basicstyle=\ttfamily\scriptsize, numbers=left,
                   frame=no, showspaces=false, showstringspaces=false,
                   caption={\textit{OpenMP} Sample Program}, captionpos=b,
                   numberstyle=\tiny,
                   xleftmargin=0.6cm,
                   label=lst:omp, keywords={%
                       \#pragma,
                       omp, parallel, firstprivate,
                       single, omp_get_num_threads,
                       for, schedule, auto,
                       critical, master,
                       NULL, int, main,
                       void, printf, return%
                   },
                   otherkeywords={::, \#pragma, \#include, <<<,>>>, \&, \*, +, -, /, [, ], >, <}
                       ]
#pragma omp parallel firstprivate(aux_dot){
  #pragma omp single
  printf("Start of parallel region,
          number of threads: %d\n",
          omp_get_num_threads ());

  #pragma omp for schedule(auto)
  for(i = 0; i < SIZE; i++){
    aux_dot += A[i] * B[i];
  }

  #pragma omp critical
  dot += aux_dot;

  #pragma omp master
  printf("Result: %d.\n", dot);
}
\end{lstlisting}
