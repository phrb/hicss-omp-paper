\section{Pthreads \& OpenMP}
\label{sec:apis}

In this section we briefly describe \textit{OpenMP} and \textit{Pthreads}.

\subsection{Pthreads}

The \textit{Pthreads} library specification was designed to provide a standard
and portable API for developing multi-threaded programs for multiple vendor
platforms.  The programming interface was specified by the \textit{IEEE POSIX
1003.1c} standard. Any implementation adhering to this standard is said to be
POSIX threads, or \textit{Pthreads}.

The C/C++ \textit{Pthreads} library controls thread execution in GNU/Linux by
creating and synchronizing threads. In this model, the execution begins in a
sequential region and threads must be explicitly started to begin
a parallel region.
Listing \ref{lst:listing-pthreads} presents a sample code using the
\textit{Pthreads} library. The program begins execution in a
single thread and until line 6 only one thread is running.

When \texttt{pthread\_create} is called a thread is created and the application
enters a parallel region with two simultaneous threads.  The main mechanism
available for synchronizing thread executions is the barrier. The
\textit{Pthreads} library implements barriers as calls to
\texttt{pthread\_join}. A thread that reaches a barrier will wait for other
threads. Line 11 shows an example of thread synchronization using a barrier.

\begin{lstlisting}[language=C, basicstyle=\ttfamily\scriptsize, numbers=left,
                   frame=no, showspaces=false, showstringspaces=false,
                   caption={\textit{Pthreads} Sample Program}, captionpos=b,
                   numberstyle=\tiny,
                   xleftmargin=0.5cm,
                   label=lst:listing-pthreads, keywords={%
                       DATATYPE, pthread_t, pthread_create,
                       pthread_join, task_function, NULL, int, main,
                       void, printf, return%
                       },
                   otherkeywords={::, \#pragma, \#include, <<<,>>>, \&, \*, +, -, /, [, ], >, <}
                   ]
int main(){
    pthread_t thread_id;
    DATATYPE *parameters = 10;
    int return_code;

    return_code = pthread_create(&thread_id, NULL,
                                 task_function,
                                 (void *) parameters);

    printf("Return Code: %d\n", return_code );
    pthread_join(thread_id , NULL);

    return 0;
}

void *task_function(void *data){
    DATATYPE *value;
    value = (DATATYPE *) data;
}
\end{lstlisting}

\subsection{OpenMP}

\textit{OpenMP} is an specification that defines a set of compiler directives,
libraries and environment variables that help exploring program parallelism.
The \textit{OpenMP} specification is kept by the \textit{OpenMP Architecture
Review Board}, composed by hardware manufacturers, and parallel software
developers and users.

\textit{OpenMP} implements the \textit{fork-join} model. An \textit{OpenMP}
program contains weaved sequential and parallel regions and always starts with
a sequential region, or master thread.  \textit{OpenMP} directives are
implemented in C/C++ and Fortran compilers. In C/C++ compilers the
specification of parallel executions is done using \textit{pragmas}, defined in
by using the \texttt{\#pragma} keyword.
The compilation process is not affected by the removal of the pragma,
which will result in the generation of a sequential version of the program.

\textit{OpenMP} directives follow the format \texttt{\#pragma directive
[clauses]}.  Each line contains at least one directive and may contain one or
more clauses. The main directives defined by the specification are related
to work sharing, task definition and dependencies, and synchronization.

Listing \ref{lst:listing-omp} presents an \textit{OpenMP} example. The
declaration of a parallel region with the \texttt{firstprivate} pragma on line
1 indicates the variable \texttt{aux\_dot} should be private to every thread
and initialized with the enclosing scope's value.  Line 3 will be executed by
only one thread, because of the \texttt{single} pragma.
The \texttt{for} in line 7 will be executed in parallel, with automatic work
division.  The pragma at line 12 defines a critical region which can only be
executed by a single thread at a time. The pragma at line 15 defines a
region that should only be executed by the master thread.

\begin{lstlisting}[language=C, basicstyle=\ttfamily\scriptsize, numbers=left,
                   frame=no, showspaces=false, showstringspaces=false,
                   caption={\textit{OpenMP} Sample Program}, captionpos=b,
                   numberstyle=\tiny,
                   xleftmargin=0.6cm,
                   label=lst:listing-omp, keywords={%
                       \#pragma,
                       omp, parallel, firstprivate,
                       single, omp_get_num_threads,
                       for, schedule, auto,
                       critical, master,
                       NULL, int, main,
                       void, printf, return%
                   },
                   otherkeywords={::, \#pragma, \#include, <<<,>>>, \&, \*, +, -, /, [, ], >, <}
                       ]
#pragma omp parallel firstprivate(aux_dot){
    #pragma omp single
    printf("Start of parallel region,
            number of threads: %d\n",
            omp_get_num_threads ());

    #pragma omp for schedule(auto)
    for(i = 0; i < SIZE; i++){
        aux_dot += A[i] * B[i];
    }

    #pragma omp critical
    dot += aux_dot;

    #pragma omp master
    printf("Result: %d.\n", dot);
}
\end{lstlisting}
